\begin{thebibliography}{9}

  \bibitem{Veiga}
  Veiga, N., Otero, L., \& Torres, J. (2020). \textit{Reflexiones sobre el uso de la estadística inferencial en investigación didáctica}. InterCambios. Dilemas y transiciones de la Educación Superior, 7(2). https://doi.org/10.29156/INTER.7.2.10. ISSN 2301-0118.

  
  \bibitem{LopezRol} 
  López-Roldán, P., \& Fachelli, S. (2016). \textit{Fundamentos de estadística inferencial}. En *Metodología de la investigación social cuantitativa* (cap. III.4). Bellaterra: Universitat Autònoma de Barcelona.

  \bibitem{CrespiMascarilla}
  Crespi-Vallbona, M., \& Mascarilla-Miró, O. (s.f.). \textit{Satisfacción laboral. El caso de los empleados del sector de las tecnologías de la información en España}. Universitat de Barcelona. Recuperado de https://ddd.uab.cat/

\end{thebibliography}