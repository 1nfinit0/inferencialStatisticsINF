\section{Conclusión}

  De acuerdo con los resultados obtenidos, tratados, analizados y discutidos en el presente trabajo, se puede concluir que los estudiantes de la Universidad Tecnológica del Perú (UTP) tienen una percepción positiva sobre el uso de herramientas de inteligencia artificial en su formación académica. La mayoría de los encuestados considera que las herramientas de inteligencia artificial influyen positivamente en su rendimiento académico, y que estas herramientas son útiles para mejorar su aprendizaje y comprensión de los temas vistos en clase. Así mismo es importante destacar que los estudiantes consideran que las herramientas de inteligencia artificial pueden ser útiles para automatizar tareas repetitivas y mejorar la calidad de sus trabajos académicos.

  Sin embargo, también se identificaron ciertos riesgos asociados con el uso de herramientas de inteligencia artificial en el ámbito académico. Los estudiantes consideran que el principal riesgo es la dependencia excesiva de estas herramientas, lo que podría limitar el desarrollo de habilidades fundamentales como el pensamiento crítico y la creatividad. Otros riesgos identificados incluyen la deshonestidad académica y la falta de desarrollo de habilidades personales.

  Es posible que la percepción positiva de los estudiantes sobre el uso de herramientas de inteligencia artificial en su formación académica se deba a la facilidad de uso y la accesibilidad de estas herramientas, así como a su capacidad para mejorar la eficiencia y la calidad del aprendizaje. Sin embargo, es importante tener en cuenta los riesgos asociados con el uso de estas herramientas y tomar medidas para mitigarlos, como fomentar el pensamiento crítico y la creatividad, y promover la responsabilidad académica.

  En resumen, los resultados obtenidos en este estudio sugieren que el uso de herramientas de inteligencia artificial en la educación puede ser beneficioso para los estudiantes, siempre y cuando se utilicen de manera responsable y se tomen medidas para abordar los riesgos asociados con su uso. Es importante seguir investigando sobre este tema para comprender mejor cómo las herramientas de inteligencia artificial pueden contribuir a mejorar la calidad de la educación y el aprendizaje.