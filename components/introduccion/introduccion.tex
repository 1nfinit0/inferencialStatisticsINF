\vspace*{\fill}

\section{Introducción}

  En el contexto actual de alta competitividad e innovación constante, las empresas del sector tecnológico enfrentan el reto de mantener la productividad de sus trabajadores en niveles óptimos. Uno de los enfoques más comunes para alcanzar este objetivo es la implementación de programas de capacitación continua, que buscan actualizar y fortalecer las habilidades del personal. Sin embargo, muchas organizaciones aún no cuentan con evidencia estadística que permita comprobar si esta inversión en formación realmente se traduce en una mejora del rendimiento laboral.
  
  Este informe se centra en el área de productividad laboral dentro del sector tecnológico, entendida como la capacidad de los trabajadores para completar tareas de manera eficiente en un determinado periodo de tiempo. Se evaluará específicamente la relación entre la cantidad de horas de capacitación continua recibidas por los empleados y su nivel de productividad. Este análisis es relevante porque, en un entorno donde el tiempo y la calidad del trabajo son claves, tomar decisiones basadas en datos puede significar una ventaja competitiva importante.
  
  El análisis estadístico inferencial permitirá explorar si existe una relación significativa entre estas dos variables, lo cual contribuirá a la toma de decisiones informadas en el diseño o ajuste de estrategias de formación. Comprender esta relación puede ayudar a las organizaciones a asignar mejor sus recursos, identificar qué tipo de capacitaciones son más efectivas y, en última instancia, mejorar su rendimiento global.

\vspace*{\fill}