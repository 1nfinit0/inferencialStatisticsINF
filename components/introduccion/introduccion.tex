% introduccion.tex
\vspace*{\fill}

\section{Introducción}

En el contexto actual de alta competitividad tecnológica, las empresas peruanas enfrentan el reto de optimizar la productividad de sus equipos. Una pregunta crítica es si los trabajadores de departamentos tecnológicos (IT) muestran niveles de desempeño significativamente diferentes a los de áreas no tecnológicas (Marketing, HR, Ventas, Finanzas), posiblemente debido a su exposición a herramientas avanzadas y entornos de innovación.

Este estudio se centra en analizar estadísticamente si existen diferencias significativas en la productividad laboral entre estos dos grupos. Utilizando datos de una empresa tecnológica peruana, evaluaremos específicamente si la media de productividad en el departamento de IT difiere de manera significativa de la observada en otros departamentos. 

Este análisis es relevante porque puede revelar patrones de desempeño asociados al contexto tecnológico, proporcionando insights valiosos para:

\begin{itemize}
  \item Asignación estratégica de recursos
  \item Diseño de programas de capacitación diferenciados
  \item Políticas de gestión del talento sectorizadas
\end{itemize}

El enfoque estadístico inferencial permitirá determinar si las diferencias observadas son significativas o atribuibles al azar, aportando evidencia empírica para la toma de decisiones.

\vspace*{\fill}