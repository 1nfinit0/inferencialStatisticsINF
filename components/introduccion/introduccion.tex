\vspace*{\fill}

\section{Introducción}

  El uso de herramientas de inteligencia artificial (IA) en la educación ha crecido significativamente, transformando las dinámicas de enseñanza y aprendizaje. En la Universidad Tecnológica del Perú (UTP), se han implementado diversas tecnologías de IA durante el periodo académico 2024 I - II, con el objetivo de mejorar el rendimiento académico de los estudiantes. Este cambio ha despertado un interés creciente por evaluar el impacto real de estas herramientas en el desempeño de los alumnos.
  
  Este proyecto tiene como objetivo realizar un análisis estadístico-descriptivo sobre el rendimiento académico de los estudiantes de la UTP que utilizan herramientas de IA. A través de la aplicación de técnicas y conceptos aprendidos en el curso de Estadística Descriptiva y Probabilidades, se espera identificar patrones y tendencias que contribuyan a una mejor comprensión del uso de estas tecnologías en el ámbito educativo y su potencial para optimizar el aprendizaje.

\vspace*{\fill}