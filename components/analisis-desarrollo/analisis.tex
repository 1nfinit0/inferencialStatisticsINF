\section{Análisis y Desarrollo}

\subsection{Estadística descriptiva}

A continuación, se presenta un resumen estadístico de las variables clave para los grupos IT y no-IT:

\begin{multicols}{2}
\textbf{IT}
\begin{itemize}
    \item \textbf{Media:}
    \[
        \bar{X}_{IT} = \frac{1}{48} \sum_{i=1}^{48} X_{i}
    \]
    \item \textbf{Desviación estándar:}
    \[
        s_{IT} = \sqrt{\frac{1}{48-1} \sum_{i=1}^{48} (X_{i} - 56.34\%)^2}
    \]
    \item \textbf{Mediana:} \\
    Valor central de los datos ordenados de $X_{1}, X_{2}, ..., X_{48}$.
\end{itemize}

\columnbreak

\textbf{No-IT}
\begin{itemize}
    \item \textbf{Media:}
    \[
        \bar{X}_{noIT} = \frac{1}{152} \sum_{j=1}^{152} Y_{j}
    \]
    \item \textbf{Desviación estándar:}
    \[
        s_{noIT} = \sqrt{\frac{1}{152-1} \sum_{j=1}^{152} (Y_{j} - 44.51\%)^2}
    \]
    \item \textbf{Mediana:} \\
    Valor central de los datos ordenados de $Y_{1}, Y_{2}, ..., Y_{152}$.
\end{itemize}
\end{multicols}

\begin{table}[H]
\centering
\begin{tabular}{|l|c|c|c|c|}
\hline
\textbf{Grupo} & \textbf{n} & \textbf{Media (\%)} & \textbf{Desv. Est. (\%)} & \textbf{Mediana (\%)} \\
\hline
IT & 48 & \textbf{56.34\%} & \textbf{30.62\%} & \textbf{62.00\%} \\
No-IT & 152 & \textbf{44.51\%} & \textbf{27.64\%} & \textbf{44.00\%} \\
\hline
\end{tabular}
\caption{Estadística descriptiva de la productividad por grupo}
\end{table}

\vspace{0.5cm}

\noindent
\textbf{Nota:} Los valores de la tabla corresponden a la variable \textit{Productivity (\%)}. Se observa que el grupo IT presenta una media de productividad de \textbf{56.34\%}, mientras que el grupo no-IT tiene una media de \textbf{44.51\%}.

\begin{figure}[H]
    \centering
    \includegraphics[width=0.7\textwidth]{assets/Boxplot.png}
    \fbox{\parbox{0.7\textwidth}{\centering \textit{[Boxplot de productividad IT vs. no-IT]}}}
    \caption{Distribución de la productividad por grupo (IT vs. no-IT)}
\end{figure}

\subsection{Prueba de hipótesis para diferencia de medias}

Se realizó una prueba t para muestras independientes bajo los siguientes parámetros:

\begin{itemize}
    \item \textbf{Hipótesis nula ($H_0$):} $\mu_{IT} = \mu_{no-IT}$
    \item \textbf{Hipótesis alternativa ($H_1$):} $\mu_{IT} \neq \mu_{no-IT}$
    \item \textbf{Nivel de significancia:} $\alpha = 0.05$
    \item \textbf{Estadístico calculado:} $t = \textbf{2.1836}$
    \item \textbf{Valor crítico:} $t_{0.025, 52.06} = \textbf{2.0066}$
    \item \textbf{p-valor:} \textbf{0.0335}
\end{itemize}

\noindent
\textbf{Decisión:} Como $|t| = 2.1836$ es mayor que el valor crítico y el p-valor es menor que $\alpha$, se \textbf{rechaza} la hipótesis nula.

\subsection{Intervalo de confianza para la diferencia de medias}

El intervalo de confianza al 95\% para la diferencia de medias es:

\[
IC_{95\%} = ([LIM\_INFERIOR],\ [LIM\_SUPERIOR])
\]

\noindent
\textbf{Interpretación:} Con un 95\% de confianza, la verdadera diferencia en productividad entre los departamentos IT y no-IT está entre \textbf{[LIM\_INFERIOR]} y \textbf{[LIM\_SUPERIOR]}. Si el intervalo incluye el cero, no hay evidencia suficiente para afirmar una diferencia significativa.

\subsection{Visualización de resultados}

\begin{figure}[H]
    \centering
    %\includegraphics[width=0.7\textwidth]{ruta/a/histograma.png}
    \fbox{\parbox{0.7\textwidth}{\centering \textit{[Histograma de productividad por grupo]}}}
    \caption{Histograma de la productividad en IT y no-IT}
\end{figure}

\subsection{Resumen de hallazgos}

\begin{itemize}
    \item La media de productividad en IT es \textbf{[MEDIA\_IT]\%}, mientras que en no-IT es \textbf{[MEDIA\_NOIT]\%}.
    \item La prueba t arroja un estadístico de \textbf{[T\_VALUE]} y un p-valor de \textbf{[P\_VALOR]}.
    \item El intervalo de confianza para la diferencia de medias es \textbf{([LIM\_INFERIOR], [LIM\_SUPERIOR])}.
    \item \textbf{[Conclusión sobre si existe o no diferencia significativa, según los resultados]}
\end{itemize}

\vspace{0.5cm}
\noindent
\textbf{Nota:} Los valores numéricos y gráficos serán reemplazados por los resultados exactos una vez completados los cálculos estadísticos.



