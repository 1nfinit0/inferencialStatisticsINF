\subsection{Recepción de la AI. en el aprendizaje a largo plazo.}

Esta tabla de frecuencia ayudará a entender cuántas personas han elegido cada opción, lo que es 
crucial para analizar la distribución de los datos. Asimismo, esto te permite ver cuántas respuestas 
caen dentro de una cierta categoría o rango, y es útil para calcular percentiles, además permite ver 
rápidamente qué categorías son más frecuentes y cuáles son menos, lo que ayuda a identificar las 
tendencias principales en los datos.

\textbf{Tabla de Frecuencias:}

\begin{table}[H]
  \centering
  \renewcommand{\arraystretch}{1.2}
  \begin{tabular}{l c c c c}
    \hline
    {Respuestas} & {\(f_i\)} & \textit{Fi} & \textit{hi}(\%) & \textit{Hi}(\%)\\
    \hline
    Preocupado  & 9  & 9  & 11.69\% & 11.69\%\\
    Neutro      & 38 & 47 & 49.35\% & 61.04\%\\
    Optimista   & 24 & 71 & 31.17\% & 92.21\%\\
    No tengo una opinión formada & 6  & 77 & 7.79\%  & 100\%\\
    \hline
    Total       & 77 &    & 100\%   & \\
    \hline
  \end{tabular}
  \caption{Distribución de respuestas sobre la recepción de la IA en el aprendizaje a largo plazo}
  \label{tabla:recepcionIA}
\end{table}

\textbf{Moda ($M_o$):} Se puede notar que la opción "Neutro" es el dato de frecuencia mayor al ser el más 
repetitivo, siendo esta la moda.

\textbf{Mediana ($M_e$):}  Contamos con un conjunto de datos con 77 elementos, por lo cual calcularemos la 
mediana con la siguiente formula:

\begin{equation*}
  M_e = \dfrac{n + 1}{2} = \dfrac{77 + 1}{2} = 39
\end{equation*}

Esto indica que la mediana se encuentra en la posición 39, la cual corresponde a la opción "Neutro".

\textbf{Interpretación de los resultados:}

La tabla muestra cómo los alumnos encuestados de la UTP durante el periodo 
2024 I-II perciben el uso de la inteligencia artificial (IA) en su aprendizaje a largo plazo. Siendo la 
categoría más "Neutro", con 38 respuestas, lo que representa el 49.35 \% de la muestra. Esto indica 
que casi la mitad de los encuestados no tiene una postura positiva ni negativa frente al impacto de 
la IA en su aprendizaje. La segunda categoría más seleccionada es "Optimista", con 24 respuestas, 
equivalente al 31.17 \%. Esto sugiere que una proporción considerable de las personas tiene una 
visión positiva sobre cómo la IA puede influir en su aprendizaje, lo que muestra una tendencia 
favorable, seguida de la opción "Preocupado" fue seleccionada por 9 personas, lo que representa 
el 11.69 \%, si bien es una minoría, indica que existe una cierta preocupación entre los encuestados, 
tal vez por posibles riesgos o limitaciones asociados con la IA. Por último, solo 6 encuestados (el 
7.79 \%) señalaron "No tengo una opinión formada", mostrando que una pequeña parte de la 
muestra no está segura o carece de información suficiente para opinar sobre el tema.


