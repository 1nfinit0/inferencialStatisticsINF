\subsection{Riesgos del uso de la I.A. en el aprendizaje}

Los encuestados consideran que el uso de las inteligencias artificiales en el aprendizaje puede traer consigo ciertos riesgos, los cuales se muestran en la siguiente tabla.

% Respuestas fi Fi hi (%) Hi (%)
% Deficiente análisis critico 1 1 1.30 % 1.30 %
% Vacío 1 2 1.30 % 2.60 %
% Dependencia excesiva 37 39 48.05 % 50.65 %
% Deshonestidad académica 9 48 32.47 % 62.34 %
% Falta de desarrollo de habilidades 
% personales 25 73 1.30 % 94.81 %
% Nada 1 74 1.30 % 96.11 %
% Poca investigación independiente 
% del alumno 1 75 1.30 % 97.41 %
% Otros 1 76 1.30 % 98.72 %
% Todo 1 77 1.30 % 100 %
% Total 77 100 %

\textbf{Tabla de Frecuencias:}

\begin{table}[H]
  \centering
  \renewcommand{\arraystretch}{1.5}
  \begin{tabular}{l c c c c}
    \hline
    {Respuestas} & {\(f_i\)} & \textit{Fi} & \textit{hi}(\%) & \textit{Hi}(\%)\\
    \hline
    Deficiente análisis crítico & 1 & 1 & 1.30\% & 1.30\%\\
    Vacío                       & 1 & 2 & 1.30\% & 2.60\%\\
    Dependencia excesiva        & 37 & 39 & 48.05\% & 50.65\%\\
    Deshonestidad académica     & 9  & 48 & 32.47\% & 62.34\%\\
    Falta de desarrollo de habilidades personales & 25 & 73 & 32.47\% & 94.81\%\\
    Nada                        & 1  & 74 & 1.30\% & 96.11\%\\
    Poca investigación independiente del alumno & 1 & 75 & 1.30\% & 97.41\%\\
    Otros                       & 1 & 76 & 1.30\% & 98.72\%\\
    Todo                        & 1 & 77 & 1.30\% & 100\%\\
    \hline
    Total                       & 77 &    & 100\% & \\
    \hline
  \end{tabular}
  \caption{Riesgos del uso de la I.A. en el aprendizaje}
  \label{tabla:riesgosIA}
\end{table}

\textbf{Moda $M_o$:}

La mayoría de los encuestados señala la opción "Dependencia excesiva", siendo esta 
respuesta la más frecuente en la muestra, con un total de 37 respuestas (48.05 \%).

\textbf{Interpretación de resultados:}

Los alumnos encuestados de la UTP (77) durante el periodo 2024 I-II identifican la "Dependencia excesiva" como el principal riesgo asociado con el uso de herramientas de inteligencia artificial (IA) en el ámbito académico. Esta opción fue seleccionada por 37 estudiantes (48.05 \%) lo que refleja una preocupación significativa sobre las habilidades para resolver problemas de manera independiente. La segunda opción más seleccionada es "Deshonestidad académica", con 9 respuestas (32.47 \%), sugiere que muchos estudiantes consideran que la IA puede facilitar prácticas como el plagio, etc. Con 25 respuestas (32.47 \%), esta categoría refleja inquietudes sobre cómo la IA podría limitar el desarrollo de habilidades fundamentales como el pensamiento crítico, la creatividad, entre otras. Finalmente, las opciones como "Deficiente análisis crítico", "Vacío", "Nada", "Poca investigación independiente del alumno", "Otros" y "Todo" apenas registran 1 respuesta cada una (1.30 \%), mostrando que estos riesgos son considerados poco relevantes por la mayoría de los encuestados.
