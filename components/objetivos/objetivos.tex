% objetivos.tex
\section{Objetivo general}

  Se pretende realizar un análisis estadístico inferencial del rendimiento laboral de trabajadores de la industria Tech en el Perú, con el fin de determinar si existen diferencias significativas en el rendimiento laboral entre aquellos que utilizan herramientas de inteligencia artificial y aquellos que no las utilizan. Para ello, se recopilarán datos sobre el rendimiento laboral de los trabajadores, así como información sobre su uso de herramientas de inteligencia artificial. Se espera que este análisis permita identificar patrones y tendencias en el rendimiento laboral, así como proporcionar recomendaciones para mejorar la productividad y eficiencia en el sector tecnológico.
  
  
  \subsection{Objetivos específicos}
  
  \begin{enumerate}
    \item \textbf{Identificar y analizar:} Se busca identificar y analizar las herramientas de inteligencia artificial más utilizadas en el sector tecnológico en Perú, así como su impacto en el rendimiento laboral de los trabajadores. Esto permitirá comprender cómo estas herramientas influyen en la productividad y eficiencia de los empleados.
    \item \textbf{Evaluar la relación:} Se pretende evaluar la relación entre el uso de herramientas de inteligencia artificial y el rendimiento laboral de los trabajadores. Esto implica analizar si existe una correlación positiva entre el uso de estas herramientas y el rendimiento laboral, así como identificar factores que puedan influir en esta relación.
    \item \textbf{Realizar un análisis estadístico:} Se llevará a cabo un análisis estadístico inferencial para determinar si existen diferencias significativas en el rendimiento laboral entre los trabajadores que utilizan herramientas de inteligencia artificial y aquellos que no las utilizan. Esto permitirá validar o refutar la hipótesis planteada y proporcionar evidencia empírica sobre el impacto de estas herramientas en el rendimiento laboral.
    \item \textbf{Proporcionar recomendaciones:} A partir de los resultados obtenidos, se buscará proporcionar recomendaciones para mejorar la productividad y eficiencia en el sector tecnológico. Esto incluirá sugerencias sobre la implementación de herramientas de inteligencia artificial, así como estrategias para fomentar su uso entre los trabajadores.
    \item \textbf{Contribuir al conocimiento:} Se espera que este análisis contribuya al conocimiento sobre el impacto de las herramientas de inteligencia artificial en el rendimiento laboral en el sector tecnológico en Perú. Esto permitirá enriquecer la literatura existente sobre el tema y proporcionar información valiosa para futuras investigaciones.
  \end{enumerate}

  Estos objetivos específicos permitirán abordar de manera integral el análisis del rendimiento laboral de los trabajadores de la industria Tech en Perú, así como su relación con el uso de herramientas de inteligencia artificial. Se espera que los resultados obtenidos sean relevantes y útiles para mejorar la productividad y eficiencia en el sector tecnológico.
