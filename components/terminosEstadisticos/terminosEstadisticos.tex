\section{Términos estadísticos}

  A continuación, se presentan los términos estadísticos que se utilizarán en el análisis del rendimiento académico de los alumnos de la UTP en relación con el uso de herramientas de inteligencia artificial. La mayor parte de las definiciones están tomadas de Gaviria y Márquez (2019), con ajustes específicos al contexto del estudio.
  
  \begin{enumerate}
    \item \textbf{Población:} Se define como el conjunto completo de elementos u objetos de interés sobre los cuales se realizarán las observaciones. En este estudio, la población estará compuesta por todos los estudiantes de la UTP que han utilizado herramientas de inteligencia artificial durante el periodo académico 2024 I - II.
  
    \item \textbf{Muestra:} Dada una población $P$, una muestra $M$ es un subconjunto representativo de dicha población. La muestra se seleccionará para obtener una representación adecuada del rendimiento académico, y se tomará de un grupo específico de estudiantes que han utilizado diversas herramientas de inteligencia artificial en su proceso de aprendizaje.
  
    \item \textbf{Unidad de análisis:} La unidad de análisis es el objeto o elemento que se observa en una investigación estadística. En este caso, la unidad de análisis será el rendimiento académico medido a través de las calificaciones de los estudiantes que han utilizado herramientas de inteligencia artificial.
  
    \item \textbf{Variable:} Una variable es una característica que puede ser medida o categorizada en una población o muestra. En este estudio, se considerarán tanto variables cualitativas, como la percepción de los estudiantes sobre el uso de herramientas de inteligencia artificial, como variables cuantitativas, tales como las calificaciones obtenidas.
  
    \item \textbf{Parámetro:} Un parámetro es un valor numérico $\theta$ que resume una característica de la población $P$. En este análisis, los parámetros se deducirán a partir del estudio del rendimiento académico en relación con el uso de herramientas de inteligencia artificial.
  
    \item \textbf{Estadístico:} Un estadístico es un valor numérico que resume la información de la muestra. También se considera una función de los datos muestrales. En este estudio, los estadísticos incluirán medidas de tendencia central, como media y mediana, y medidas de dispersión, como desviación estándar, que se calcularán a partir de las calificaciones de los estudiantes.
\end{enumerate}