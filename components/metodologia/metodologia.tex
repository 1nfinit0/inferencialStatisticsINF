\section{Metodología}

	\subsection{Tipo de muestreo}
	
	Para la selección de la muestra se utilizará un muestreo aleatorio simple, donde cada trabajador del sector tecnológico que haya respondido la encuesta será seleccionado de forma independiente. Este método permite que todos los individuos tengan la misma probabilidad de ser elegidos, garantizando así la objetividad en la recolección de datos. Se establecerá el tamaño de la muestra en función del número total de respuestas válidas obtenidas. Además, se considerarán criterios de inclusión, como tener al menos seis meses de experiencia en la empresa y haber participado en programas de capacitación durante el último año, y criterios de exclusión, como trabajadores en periodo de prueba o en licencia prolongada. El objetivo es asegurar una muestra representativa de la población activa y capacitada dentro del sector tecnológico.
	
	\subsection{Técnicas de análisis}
	
	El análisis de los datos se realizará utilizando técnicas estadísticas descriptivas e inferenciales. Inicialmente, se emplearán medidas de tendencia central (media, mediana) y de dispersión (desviación estándar) para describir las variables principales: cantidad de horas de capacitación continua y niveles de productividad. También se presentarán los datos mediante tablas de frecuencia y gráficos que permitan una visualización clara de los resultados.
	
	Posteriormente, se aplicarán técnicas de estadística inferencial. Se calcularán intervalos de confianza para las medias de ambas variables y se realizará una prueba de hipótesis para determinar si existe una relación significativa entre la capacitación continua y la productividad. En particular, se utilizará el coeficiente de correlación de Pearson, siempre que los datos cumplan con el supuesto de normalidad. Estas herramientas permitirán evaluar la dirección e intensidad del vínculo entre las variables estudiadas.
	
	\subsection{Consideraciones éticas}
	
	Durante la recolección y análisis de los datos se garantizará la confidencialidad y el anonimato de los participantes. Los trabajadores serán informados sobre los fines académicos del estudio y se solicitará su consentimiento antes de responder la encuesta. Ninguna información personal será divulgada, y los resultados se presentarán de forma agregada, evitando cualquier posibilidad de identificación individual.
	
	\subsection{Limitaciones del estudio}
	
	Entre las limitaciones del presente estudio se encuentra la posibilidad de sesgos en las respuestas autodeclaradas por los trabajadores, especialmente en lo relativo a su percepción de productividad. Asimismo, al tratarse de una muestra no vinculada a una empresa específica, los resultados podrían no ser generalizables a todas las organizaciones tecnológicas. También existen variables externas no controladas, como el entorno laboral o el liderazgo de los equipos, que podrían influir en los niveles de productividad y no se contemplan en este análisis.
