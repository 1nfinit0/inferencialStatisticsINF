% metodologia.tex
\section{Metodología}

	\subsection{Enfoque estadístico}
	El estudio emplea técnicas de estadística inferencial para contrastar la hipótesis de diferencia de medias entre grupos independientes (departamento IT vs. no-IT), siguiendo los procedimientos establecidos en el temario:

	\begin{itemize}
		\item \textbf{Tipo de muestreo:} Aleatorio simple (asumiendo representatividad de la muestra disponible)
		\item \textbf{Población:} Trabajadores de empresas tecnológicas peruanas
		\item \textbf{Muestra:} 200 empleados (n\textsubscript{IT}=48, n\textsubscript{no-IT}=152)
		\item \textbf{Nivel de confianza:} 95\% ($\alpha$=0.05)
	\end{itemize}

	\subsection{Prueba de hipótesis para diferencia de medias}
	Se aplicará el procedimiento de contraste de hipótesis para dos medias con varianzas poblacionales desconocidas, según lo cubierto en el temario:

	\begin{align*}
		H_0: \mu_{IT} &= \mu_{no-IT} \\
		H_1: \mu_{IT} &\neq \mu_{no-IT}
	\end{align*}

		\subsubsection*{Justificación técnica}
		\begin{itemize}
			\item \textbf{Distribución t de Student:} Aplicable por desconocerse las varianzas poblacionales
			\item \textbf{Teorema del Límite Central:} Valida la normalidad aproximada con n > 30 por grupo
			\item \textbf{Fórmula del estadístico:} 
			$t = \frac{(\bar{X}_1 - \bar{X}_2)}{\sqrt{\frac{s_1^2}{n_1} + \frac{s_2^2}{n_2}}}$
			\item \textbf{Regla de decisión:} Rechazar $H_0$ si $|t| > t_{\alpha/2, gl}$
		\end{itemize}

	\subsection{Intervalos de confianza}
	Se calculará el intervalo para la diferencia de medias según el temario:

		\subsubsection*{Fórmula e interpretación}
		\begin{itemize}
			\item \textbf{Fórmula:} $IC_{95\%} = (\bar{X}_1 - \bar{X}_2) \pm t_{\alpha/2, gl} \cdot \sqrt{\frac{s_1^2}{n_1} + \frac{s_2^2}{n_2}}$
			\item \textbf{Interpretación:} "Con 95\% de confianza, la verdadera diferencia en productividad entre departamentos IT y no-IT está entre [LÍMITE INFERIOR] y [LÍMITE SUPERIOR]"
		\end{itemize}

	\subsection{Análisis complementarios}
	\begin{itemize}
		\item \textbf{Tablas estadísticas:} Uso de tablas t-Student para valores críticos
		\item \textbf{Análisis descriptivo:} Medidas de tendencia central y dispersión por grupo
	\end{itemize}

	\begin{table}[H]
		\centering
		\begin{tabular}{|l|c|c|}
		\hline
		\textbf{Elemento} & \textbf{Simbología} & \textbf{Valor} \\ \hline
		Nivel de significancia & $\alpha$ & 0.05 \\ \hline
		Grados de libertad & $gl$ & $n_1 + n_2 - 2$ \\ \hline
		Valor crítico bilateral & $t_{\alpha/2, gl}$ & $\pm 1.96$ \\ \hline
		Error estándar & $SE$ & $\sqrt{\frac{s_1^2}{n_1} + \frac{s_2^2}{n_2}}$ \\ \hline
		\end{tabular}
		\caption{Parámetros estadísticos según temario}
	\end{table}