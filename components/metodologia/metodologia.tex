\section{Metodología}

  \subsection{Tipo de muestreo}

  Para la selección de la muestra, se utilizará un muestreo aleatorio simple, donde cada estudiante de la UTP que haya realizado la encuesta será seleccionado de forma independiente. Se establece un tamaño de muestra de acuerdo a la cantidad de respuestas, esto para garantizar la representatividad de los datos y se considerarán criterios de inclusión y exclusión para definir el grupo de participantes. Se espera obtener una muestra diversa y equilibrada que refleje la heterogeneidad de la población estudiantil de la UTP.

  \subsection{Técnicas de análisis}

  El análisis de los datos recolectados se realizará mediante técnicas estadísticas descriptivas, que permitirán resumir y visualizar la información obtenida. Se calcularán medidas de tendencia central, como la media, mediana y moda, así como medidas de dispersión, como la desviación estándar, para analizar el rendimiento académico de los estudiantes en relación con el uso de herramientas de inteligencia artificial. Además, se emplearán gráficos y tablas de frecuencias para representar los resultados de manera clara y comprensible.

  \subsection{Consideraciones éticas}

  Durante la recolección de información, se garantizará la confidencialidad y privacidad de los datos de los estudiantes, evitando la divulgación de información personal sin su consentimiento. Se solicitará la autorización de los participantes para el uso de sus respuestas con fines académicos y se asegurará que los resultados del estudio se presenten de manera anónima y agregada, sin identificar a los individuos de forma individual.

  \subsection{Limitaciones del estudio}

  Algunas limitaciones que pueden afectar la validez y generalización de los resultados incluyen el tamaño de la muestra, que podría no ser representativo de toda la población estudiantil de la UTP, así como la posibilidad de sesgos en las respuestas de los participantes. Además, la falta de control sobre variables externas que puedan influir en el rendimiento académico de los estudiantes, como el nivel de motivación o el entorno familiar, podría limitar la interpretación de los resultados.