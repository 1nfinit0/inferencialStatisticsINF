\section{Marco Teórico}

  \subsection{La estadística inferencial}

  La estadística inferencial es una rama de la estadística que se encarga de hacer inferencias o generalizaciones sobre una población a partir de una muestra representativa. A través de técnicas estadísticas, se pueden estimar parámetros poblacionales, realizar pruebas de hipótesis y calcular intervalos de confianza. La estadística inferencial es fundamental en la investigación científica, ya que permite tomar decisiones basadas en datos y evaluar la validez de las afirmaciones realizadas a partir de muestras.

  \begin{flushright}
    \textit{El análisis estadístico inferencial provee herramientas que 
    permiten la evaluación sistemática y eficiente de una muestra de la población que se quiere estudiar.}
    \cite{Veiga}
  \end{flushright}
  \begin{flushright}
    \textit{Veiga, Otero, Torres (2020)}
  \end{flushright}

  A fin de elaborar un correcto marco teórico se consideró pertienente consultar las fuentes bibliográficas adecuadas. De esta manera Lopez y Fachelli. (2016) nos dice que la \textit{Estadística Inferencial} es aquella que se encarga de realizar inferencias sobre una población a partir de una muestra. Esto implica que, a partir de los datos obtenidos de una muestra, se pueden hacer afirmaciones o generalizaciones sobre la población de la cual se extrajo la muestra.

  \subsection{El ambiente laboral tenológico}

  El sector tecnológico se caracteriza por su dinamismo y constante evolución, lo que plantea desafíos y oportunidades para los trabajadores. La capacitación continua se ha convertido en un elemento clave para mantener la competitividad en este entorno. Las empresas tecnológicas deben adaptarse rápidamente a los cambios en la industria, lo que requiere que sus empleados estén actualizados en las últimas tendencias y tecnologías.

  Según Vallbona y Mascarilla. (2018) Nos dice: Los empleados del sector de las Tecnologías de la Información en España manifiestan, en general, niveles elevados de satisfacción laboral, especialmente en lo referente a la autonomía en el trabajo y las oportunidades de desarrollo profesional

  \begin{flushright}
    \textit{Satisfacción laboral. El caso de los empleados del sector de las tecnologías de la información en España.}
    \cite{CrespiMascarilla}
  \end{flushright}
  \begin{flushright}
    \textit{Vallbona, Mascarilla (2018)}
  \end{flushright}
