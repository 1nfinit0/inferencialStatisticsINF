% planteamientoHipotesis.tex
\section{Planteamiento de Hipótesis}

El presente estudio tiene como objetivo evaluar si existen diferencias significativas en el nivel de productividad entre los trabajadores de departamentos tecnológicos (específicamente el departamento de Tecnologías de la Información, IT) y aquellos de departamentos no tecnológicos (como Marketing, Recursos Humanos, Ventas y Finanzas). Esta comparación permite explorar si el contexto laboral tecnológico, donde es más probable el uso de herramientas avanzadas como la inteligencia artificial, está asociado a un desempeño diferenciado. Para ello, se plantea una hipótesis estadística que será evaluada mediante una prueba de diferencia de medias.

  \subsection*{Hipótesis nula y alternativa}

  Se parte de la premisa de que el departamento de IT, al estar más expuesto a tecnologías innovadoras, podría mostrar niveles de productividad distintos. Sin embargo, esta afirmación requiere validación estadística. Con base en ello, se formulan las siguientes hipótesis:

  \begin{itemize}
    \item \textbf{Hipótesis nula (H\textsubscript{0}):} $\mu_{\text{IT}} = \mu_{\text{No\_IT}}$ (No existe diferencia significativa en la productividad media entre departamentos tecnológicos y no tecnológicos).
    \item \textbf{Hipótesis alternativa (H\textsubscript{1}):} $\mu_{\text{IT}} \neq \mu_{\text{No\_IT}}$ (Existe una diferencia significativa en la productividad media entre ambos grupos de departamentos).
  \end{itemize}

  \subsection*{Justificación}

  Esta hipótesis se alinea con los objetivos del estudio de evaluar el impacto de herramientas tecnológicas en el rendimiento laboral, considerando que:
  \begin{enumerate}
    \item El departamento de IT opera como \textit{proxy} para trabajadores con mayor exposición a tecnologías avanzadas (incluyendo IA).
    \item Permite responder a la pregunta: ¿El entorno tecnológico está asociado a un mejor desempeño?
    \item La prueba estadística seleccionada (prueba t para muestras independientes) es adecuada porque:
      \begin{itemize}
        \item La varianza poblacional es desconocida
        \item Las muestras son independientes (distintos departamentos)
        \item La variable dependiente (productividad) es cuantitativa continua
      \end{itemize}
  \end{enumerate}

  \subsection*{Variables clave}
  \begin{itemize}
    \item \textbf{Variable independiente:} Tipo de departamento (categórica: IT vs. No IT)
    \item \textbf{Variable dependiente:} Porcentaje de productividad (cuantitativa continua)
    \item \textbf{Variables de control:} 
      \begin{itemize}
        \item Posición jerárquica (Manager, Analyst, etc.)
        \item Años de experiencia (derivados de la fecha de ingreso)
        \item Proyectos completados
      \end{itemize}
  \end{itemize}