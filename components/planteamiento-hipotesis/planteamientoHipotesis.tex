\section{Planteamiento de Hipotesis}

	El presente estudio busca analizar la relación entre la capacitación continua y la productividad laboral en el sector tecnológico. La variable independiente es la capacitación continua, entendida como el número de horas de formación técnica o profesional recibidas por los trabajadores durante un periodo determinado. La variable dependiente es la productividad, medida en términos de cumplimiento de metas, número de tareas completadas o eficiencia en la ejecución de funciones.
	
	Con base en observaciones generales en el ámbito tecnológico y en teorías de gestión del talento humano, se plantea la siguiente hipótesis de investigación:
	
	Hipótesis nula $(H_0)$: No existe una relación significativa entre la capacitación continua de los trabajadores y su nivel de productividad.
	Hipótesis alternativa $(H_1)$: Existe una relación significativa entre la capacitación continua de los trabajadores y su nivel de productividad.
	
	Esta hipótesis surge a partir de la necesidad de validar, mediante herramientas estadísticas, si las capacitaciones representan una inversión estratégica que impacta positivamente en el rendimiento de los colaboradores. El análisis inferencial permitirá determinar si los resultados observados en la muestra pueden generalizarse a la población de trabajadores del sector, y si la relación entre ambas variables es estadísticamente significativa o producto del azar. Este enfoque contribuirá a una toma de decisiones basada en evidencia para optimizar los procesos de desarrollo organizacional.