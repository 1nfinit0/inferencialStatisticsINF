\section{Planteamiento de Hipotesis}

El presente estudio tiene como objetivo evaluar si el nivel promedio de productividad de los trabajadores en el sector tecnológico difiere significativamente de un valor de referencia establecido por la empresa, el cual se considera óptimo para un rendimiento sostenible. Para ello, se plantea una hipótesis estadística que permitirá verificar esta afirmación mediante una prueba de hipótesis para la media, asumiendo que la varianza de la población es conocida y utilizando la distribución Z.

\subsection*{Hipótesis nula y alternativa}

La empresa ha establecido que una productividad del 50% representa un nivel adecuado de desempeño entre los empleados, en función de los indicadores internos y los estándares de la industria. Con base en ello, se formulan las siguientes hipótesis:

\begin{itemize}
\item \textbf{Hipótesis nula (H\textsubscript{0}):} $\mu = 50$
\item \textbf{Hipótesis alternativa (H\textsubscript{1}):} $\mu \neq 50$
\end{itemize}

Donde:
\begin{itemize}
\item $\mu$ representa la media poblacional del porcentaje de productividad.
\end{itemize}

\subsection*{Justificación}

Esta hipótesis permitirá determinar si existen evidencias suficientes para afirmar que la productividad real de los trabajadores del sector tecnológico se desvía significativamente del estándar empresarial. El contraste de hipótesis se realizará mediante una prueba bilateral, ya que se desea identificar cualquier desviación —sea superior o inferior— respecto al valor esperado de 50%.

El análisis se desarrollará bajo el supuesto de que la varianza poblacional es conocida, lo que permite aplicar una prueba Z sobre la media muestral. Esta elección se justifica por la disponibilidad de una muestra suficientemente grande y por tratarse de una variable continua con comportamiento aproximadamente normal, lo que cumple los requisitos para utilizar la distribución normal estándar.

\subsection*{Variables clave}

\begin{itemize}
\item \textbf{Variable independiente:} No aplica directamente en este diseño, ya que se analiza una sola media poblacional.
\item \textbf{Variable dependiente:} Porcentaje de productividad \(\%\).
\end{itemize}